\documentclass[12pt, a4paper,oneside,leqno]{article} 
\setlength{\oddsidemargin}{0.2 in}
%\setlength{\evensidemargin}{0 in}
\setlength{\topmargin}{-1 in}
\setlength{\textwidth}{6 in}
\setlength{\textheight}{9.5 in}
\setlength{\headsep}{0.75 in}
\setlength{\parindent}{0 in}
\setlength{\parskip}{0.1 in}
\usepackage{array}
\usepackage{fancybox}
%\usepackage{unicode-math}
\usepackage{hyperref}
\hypersetup{
    colorlinks=true, % make the links colored
    linkcolor=blue, % color TOC links in blue
    urlcolor=cyan, % color URLs in red
    linktoc=all % 'all' will create links for everything in the TOC
}
\usepackage{nameref}
\usepackage{fancyhdr}
\pagestyle{fancy}
\fancyhf{}
\rhead{  \thepage \hspace{-1em} \vspace{-4em}}
\lhead{ \itshape \hspace{-1.3em}  \textsf{ HW\_8\_Solution} \vspace{-4em}}

\renewcommand{\headrulewidth}{0pt}
\usepackage{color}
\usepackage[svgnames]{xcolor}
\definecolor{dkgreen}{rgb}{0,0.6,0}
\definecolor{gray}{rgb}{0.5,0.5,0.5}
\definecolor{mauve}{rgb}{0.8,0.188,0.48}
\usepackage{mdframed}
\usepackage{framed}
%\usepackage{amsfonts}
\usepackage{soul}
\definecolor{shadecolor}{rgb}{0.92,0.92,0.92}
\definecolor{bluecolor}{rgb}{0.2,0.2,0.99}
%\usepackage{mathrsfs}
\usepackage{amsbsy}
\usepackage{setspace}
\usepackage{listings}
\usepackage{enumitem}  
%\usepackage{everysel}
\fontdimen7\font=5em
%\usepackage{type1ec}
%\usepackage{type1cm}
%\usepackage{mathabx}
%\usepackage[scale=0.9]{fdsymbol}
%\usepackage[sloped]{fourier}
%\usepackage{libertine}
\usepackage{booktabs} %调整表格线与上下内容的间隔
\usepackage{multirow}
\usepackage[most]{tcolorbox}
%\usepackage{tgbonum}
%\usepackage[bitstream-charter,cal=cmcal]{mathdesign}
\usepackage[T1]{fontenc} % 与tgbonum一起
\usepackage[upint,lcgreekalpha]{stix2}
\DeclareFontFamily{TU}{mdbch}{}
\DeclareFontShape{TU}{mdbch}{m}{n}{
 <-> \UnicodeFontFile{lmroman10-regular}{\UnicodeFontTeXLigatures}
}{}
\usepackage{amsmath}
\usepackage{bm}
\usepackage{mathtools}
\usepackage[nameinlink,capitalize]{cleveref}

\usepackage{amsthm}
%\usepackage{amssymb}
%\usepackage{graphics}
\usepackage[english]{babel}
\usepackage{graphicx}
\usepackage{float}
\usepackage{subfigure}
%\usepackage{subcaption}
\usepackage{float}
%\usepackage{inputenc}
\usepackage{xifthen}
\usepackage{xeCJK}
\newtcolorbox{definition}{
freelance,
breakable,
before=\par\vspace{2\bigskipamount}\noindent,
after=\par\bigskip,
frame code={
  \node[
  anchor=south west,
  inner xsep=8pt,
  xshift=8pt,
  rounded corners=5pt,
  font=\bfseries\color{white},
  fill=gray] at (frame.north west) (tit) {\strut definition:};
  \draw[
  line width=3pt,
  rounded corners=5pt,gray
  ] (tit.west) -| (frame.south west) -- ([xshift=15pt]frame.south west);
},
interior code={},
top=2pt
}
% the following commands set up the lecnum (lecture number)
% counter and make various numbering schemes work relative
% to the lecture number.
%
%\newtheorem{prototheorem}{Theorem}[section]

%\newenvironment{theorem}
 %  {\colorlet{shadecolor}{orange!15}\begin{shaded}\begin{prototheorem}}
  % {\end{prototheorem}\end{shaded}}






\newtheorem{theorem}{Theorem}
\newtheorem{lemma}[theorem]{Lemma}
\newtheorem{proposition}[theorem]{Proposition}
\newtheorem{claim}[theorem]{Claim}
\newtheorem{corollary}[theorem]{Corollary}
\newtheorem{fact}[theorem]{Fact}
\newtheorem{assumption}{Assumption}
\newtheorem{conjecture}[theorem]{Conjecture}
\newtheorem{observation}[theorem]{Observation}

\renewenvironment{proof}{\noindent{ Proof.}\hspace*{1em}}{\qed\vspace{1em}\\}

\newenvironment{proof-sketch}{\noindent{ Sketch of Proof:}
  \hspace*{1em}}{\qed\bigskip\\}
\newenvironment{proof-idea}{\noindent{ Proof Idea}
  \hspace*{1em}}{\qed\bigskip\\}
\newenvironment{proof-of-lemma}[1][{}]{\noindent{ Proof of Lemma {#1}}
  \hspace*{1em}}{\qed\bigskip\\}
\newenvironment{proof-of-proposition}[1][{}]{\noindent{
    Proof of Proposition {#1}}
  \hspace*{1em}}{\qed\bigskip\\}
\newenvironment{proof-of-theorem}[1][{}]{\noindent{ Proof of Theorem {#1}}
  \hspace*{1em}}{\qed\bigskip\\}
\newenvironment{inner-proof}{\noindent{ Proof}\hspace{1em}}{
  $\bigtriangledown$\medskip\\}
  \newenvironment{proof-attempt}{\noindent{ Proof attempt}
  \hspace*{1em}}{\qed\bigskip\\}
\newenvironment{proofof}[1]{\noindent{ Proof} of {#1}:
  \hspace*{1em}}{\qed\bigskip\\}
  \newenvironment{remark}{\noindent{\bfseries Remark. \hspace{-1.5em}}
  \hspace*{1em}}{\bigskip}
\lstset{ 
  frame=single,
  language=R,                     % the language of the code
  aboveskip = 3mm,
  belowskip = 2mm,
  basicstyle=\footnotesize\ttfamily, % the size of the fonts that are used for the code
  %numbers=left, % where to put the line-numbers
  columns = flexible,
  numbers = left,
  numberstyle = \tiny\color{gray},
  stepnumber=1,                   % the step between two line-numbers. If it is 1, each line
                                  % will be numbered
  numbersep=5pt,                  % how far the line-numbers are from the code
  backgroundcolor=\color[gray]{0.93},  % choose the background color. You must add \usepackage{color}
  showspaces=false,               % show spaces adding particular underscores
  showstringspaces=false,         % underline spaces within strings
  showtabs=false,                 % show tabs within strings adding particular underscores
  %frame=single,                   % adds a frame around the code
  rulecolor=\color{black},        % if not set, the frame-color may be changed on line-breaks within not-black text (e.g. commens (green here))
  tabsize=3,                      % sets default tabsize to 2 spaces
  captionpos=b,                   % sets the caption-position to bottom
  breaklines=true,                % sets automatic line breaking
  breakatwhitespace=false,        % sets if automatic breaks should only happen at whitespace
  keywordstyle = \color{RoyalBlue},
  commentstyle=\color{dkgreen},   % comment style
  stringstyle=\color{ForestGreen}      % string literal style
} 

\newcounter{example}

\newenvironment{example}[1][]{
  \refstepcounter{example}
  \ifthenelse{\isempty{#1}}{%
    % there is no text to display in the example
    \noindent \textbf{Example \theexample:}\hspace*{.05em}
  }{%
    \noindent \textbf{Example \theexample} ({#1})\textbf{:}\hspace*{.05em}
  }
}{%
  $\clubsuit$ \bigskip
}

\newenvironment{example*}[1][]{
  \ifthenelse{\isempty{#1}}{%
    % there is no text to display in the example
    \noindent \textbf{Example:}\hspace*{.05em}
  }{%
    \noindent \textbf{Example} ({#1})\textbf{:}\hspace*{.05em}
  }
}{%
  $\clubsuit$ \bigskip
}

\renewcommand{\cite}[1]{[#1]}
\def\beginrefs{\begin{list}%
        {[\arabic{equation}]}{\usecounter{equation}
         \setlength{\leftmargin}{2.0truecm}\setlength{\labelsep}{0.4truecm}%
         \setlength{\labelwidth}{1.6truecm}}}
\def\endrefs{\end{list}}
\def\bibentry#1{\item[\hbox{[#1]}]}
%usage: \fig{number}{space-in-inches}{caption}
\newcommand{\fig}[3]{
			\vspace{#2}
			\begin{center}
			figure \thelecnum.#1:~#3
			\end{center}
	}
% use these for theorems, lemmas, proofs, etc.


\newcommand{\lecture}[4]{
\begin{center}
  \hspace*{-0.7em}\fcolorbox{black}{white}{
    \begin{minipage}{1.011\columnwidth}
        \text{\bfseries \large #1}
      \hfill
      \text{\bfseries \large Spring 2020} \\
      \vspace{-.35cm}

      \begin{center}
        {\Large #2: Suggested Solutions} \\
      \end{center}
      \textsf{\itshape \normalsize Instructor: \href{mailto: yhu@jhu.edu}{#3}}
      \hfill
      \textsf{\itshape  By: \href{mailto: tzhou11@jhu.edu}{#4}  }  % You should put your name here if you scribe
    \end{minipage}
  }
\end{center}
}
%\title{}
%\author{Tong Zhou}
\usepackage{blindtext}
%\date{\today}
\begin{document}
\thispagestyle{empty}
%\lecture{Homework}{3}{Yingyao Hu}{Tong Zhou}
\lecture{AS.180.633: Econometrics}{Homework 8}{ Yingyao Hu}{ Tong Zhou}
\small
\begin{spacing}{1.3}
  \fontdimen2\font=0.4em
{\colorlet{shadecolor}{blue!10}\begin{shaded}\vspace{-2em}
    \section*{\large \text{11.14}}   \vspace{-1.5em}
    JHU \dots
Take the model 
\begin{align*}
y_{i} &=  \bm{\pi}_{i}^\prime \bm{\beta} + e_{i} \\
\bm{\pi}_{i} &= \mathbb{E}[\bm{x}_{i} | \bm{z} _{i} ] = \bm{\Gamma}^\prime \bm{z} _{i} \\
\mathbb{E}[e_{i} | \bm{z}_{i} ] &=  0
\end{align*}
where $y_{i}$ is scalar, $\bm{x}_{i} $ is a $k$ vector and  $\bm{z}_{i} $ is an $\ell$ vector.  $\bm{\beta} $ and $\bm{\pi}_{i} $ are $k \times  1$ and $\bm{\Gamma} $ is $\ell \times k$. The sample is $\left( y_{i}, \bm{x}_{i}, \bm{z}_{i} \colon i = 1, \cdots, n   \right)$ with $\bm{\pi}_{i} $ unobserved.

Consider the estimator $\widehat{\bm{\beta} } $ for $\bm{\beta} $ by OLS of $y_{i}$ on $\widehat{\bm{\pi} }_{i} = \widehat{\bm{\Gamma} }^\prime \bm{z}_{i}  $, where $\widehat{\bm{\Gamma} } $ is the OLS coefficient from the multivariate regression of $\bm{x}_{i} $ on $\bm{z}_{i} $. 
\begin{enumerate}[label = (\alph*)]
  \item Show that $\widehat{\bm{\beta} } $ is consistent for $\bm{\beta} $.
  \item Find the asymptotic distribution $\sqrt{n}\left( \widehat{\bm{\beta} } - \bm{\beta}   \right) $ as $n       \to \infty$, assuming that $\bm{\beta}=\bm{0}  $ 
  \item Why is the assumption $\bm{\beta}=\bm{0}  $ an important simplifying condition in part (b)?
  \item Using the result in (c), construct an appropriate asymptotic test for the hypothesis $\mathbb{H}_{0}\colon \bm{\beta}=\bm{0}  $.
\end{enumerate}
\end{shaded}}
\vspace{-1.5em}
  
{\colorlet{shadecolor}{gray!15}\begin{shaded}
   \begin{proof}
     \textbf{(a)}

     Since $\bm{y} = \bm{Z} \bm{\Gamma} \bm{\beta} + \bm{e}     $,
     \begin{align*}
       \widehat{\bm{\beta} } &= \left( \widehat{\bm{\Gamma} }^\prime \bm{Z}^\prime \bm{Z} \widehat{\bm{\Gamma} }     \right)^{-1} \widehat{\bm{\Gamma} }^\prime \bm{Z}^\prime \bm{y}   \\
       &=  \left( \widehat{\bm{\Gamma} }^\prime \bm{Z}^\prime \bm{Z} \widehat{\bm{\Gamma} }     \right)^{-1} \widehat{\bm{\Gamma} }^\prime \bm{Z}^\prime \bm{Z} \bm{\Gamma} \bm{\beta} +  \left( \widehat{\bm{\Gamma} }^\prime \bm{Z}^\prime \bm{Z} \widehat{\bm{\Gamma} }     \right)^{-1} \widehat{\bm{\Gamma} }^\prime \bm{Z}^\prime \bm{e}    \\
       &=  \left( \widehat{\bm{\Gamma} }^\prime \left(\frac{1}{n}\bm{Z}^\prime \bm{Z}\right) \widehat{\bm{\Gamma} }     \right)^{-1} \widehat{\bm{\Gamma} }^\prime \left(\frac{1}{n}\bm{Z}^\prime \bm{Z}\right) \bm{\Gamma} \bm{\beta} +  \left( \widehat{\bm{\Gamma} }^\prime \left(\frac{1}{n}\bm{Z}^\prime \bm{Z}\right) \widehat{\bm{\Gamma} }     \right)^{-1} \widehat{\bm{\Gamma} }^\prime (\frac{1}{n}\bm{Z}^\prime \bm{e}).   
     \end{align*}
       By $\mathbb{E}[\bm{x}_{i} | \bm{z}_{i}  ] = \bm{\Gamma}^\prime \bm{z}_{i}  $, $\widehat{\bm{\Gamma} } \stackrel{\mathsf{P}}{\longrightarrow} \bm{\Gamma}  $. Also, $\frac{1}{n}\bm{Z}^\prime \bm{Z} \stackrel{\mathsf{P}}{\longrightarrow} \bm{Q}_{zz}\equiv \mathbb{E}[\bm{z}_{i}\bm{z}_{i}^\prime  ]   $ and $ \frac{1}{n}\bm{Z}^\prime \bm{e} \stackrel{\mathsf{P}}{\longrightarrow} \bm{0}   $ by $\mathbb{E}[e_{i}|\bm{z}_{i} ]=0$.

       Therefore, we find \[
         \widehat{\bm{\beta} } \stackrel{\mathsf{P}}{\longrightarrow} \left( \bm{\Gamma}^\prime \bm{Q}_{zz} \bm{\Gamma}    \right)^{-1} \left( \bm{\Gamma}^\prime \bm{Q}_{zz} \bm{\Gamma}    \right) \bm{\beta} \equiv \bm{\beta}   
       \]
   \textbf{(b)}

       Following (a), we have 
\begin{align}
  \sqrt{n}\left( \widehat{\bm{\beta} } - \bm{\beta}   \right) &=  \left( \widehat{\bm{\Gamma} }^\prime \left( \frac{1}{n}\bm{Z}^\prime \bm{Z}   \right)^{-1} \widehat{\bm{\Gamma} }   \right)^{-1} \widehat{\bm{\Gamma} }^\prime \left( \frac{1}{n} \bm{Z}^\prime \bm{Z}   \right)\bm{\Gamma}\bm{\beta} - \bm{\beta} \label{term1} \\
  &+ \left( \widehat{\bm{\Gamma} }^\prime \left( \frac{1}{n}\bm{Z}^\prime\bm{Z}   \right)\widehat{\Gamma}   \right)^{-1} \widehat{\bm{\Gamma} }^\prime \left( \frac{1}{\sqrt{n} } \bm{Z}^\prime \bm{e}  \right) 
\end{align}         

Under the assumption $\bm{\beta} = \bm{0}  $, it reduces to 
\begin{align*}
  \sqrt{n}\left( \widehat{\bm{\beta} } - \bm{\beta}   \right) =  \left( \widehat{\bm{\Gamma} }^\prime \left( \frac{1}{n}\bm{Z}^\prime\bm{Z}   \right)\widehat{\Gamma}   \right)^{-1} \widehat{\bm{\Gamma} }^\prime \left( \frac{1}{\sqrt{n} } \bm{Z}^\prime \bm{e}  \right) 
\end{align*}

By the CLT, $\frac{1}{\sqrt{n} } \bm{Z}^\prime \bm{e} \stackrel{\mathsf{d}}{\longrightarrow} \mathscr{N}\left( \bm{0}, \Omega  \right)   $, where $\Omega = \mathbb{E}[\bm{z}_{i}\bm{z}_{i}^\prime e_{i}^2  ].$ Hence,\[
  \sqrt{n}\left( \widehat{\bm{\beta} } - \bm{\beta}   \right) \stackrel{\mathsf{d}}{\longrightarrow} \mathscr{N}\left( \bm{0}, \left( \bm{\Gamma}^\prime \bm{Q}_{zz} \bm{\Gamma}    \right)^{-1} \left( \bm{\Gamma}^\prime \bm{\Omega} \bm{\Gamma}    \right) \left( \bm{\Gamma}^\prime \bm{Q}_{zz} \bm{\Gamma}    \right)^{-1}  \right)  
\]
\textbf{(c)}

Without $\bm{\beta} = \bm{0}  $, the term \cref{term1} is $O_{p}(1)$ and therefore contaminates the asymptotic distribution in (b)

\textbf{(d)}

Under $\bm{\beta} = \bm{0}  $, apply the result in (b), a Wald statistic can be used \[
  \mathcal{W}_{n} = n  \widehat{\bm{\beta} } ^{\,^\prime}  \widehat{\bm{V} }_{\bm{\beta} }^{-1} \widehat{\bm{\beta} }      
\]
where $\widehat{\bm{V} }_{\bm{\beta} } = \left( \widehat{\bm{\Gamma} }^\prime \widehat{\bm{Q} }_{zz} \widehat{\bm{\Gamma} }    \right)^{-1} \left( \widehat{\bm{\Gamma} }^\prime \widehat{\bm{\Omega} } \widehat{\bm{\Gamma} }    \right)\left( \widehat{\bm{\Gamma} }^\prime \widehat{\bm{Q} }_{zz} \widehat{\bm{\Gamma} }    \right)^{-1}. $
     \end{proof}\vspace{-1.5em}
\end{shaded}}


{\colorlet{shadecolor}{blue!10}\begin{shaded}\vspace{-2em}
    \section*{\large \text{12.2}}   \vspace{-1.5em}
In the linear model 
\begin{align*}
y_{i} &= \bm{x}_{i}^\prime \bm{\beta} + e_{i} \\
\mathbb{E}[e_{i} | \bm{x}_{i} ] &= 0
\end{align*}
suppose $\sigma_{i}^2 = \mathbb{E}[e_{i}^2 | x_{i}]$ is known. Show that the GLS estimator of $\bm{\beta} $ can be written as an IV estimator using some instrument $\bm{z}_{i} $. (Find an expression for $\bm{z}_{i} $.)
\end{shaded}}

       {\colorlet{shadecolor}{gray!15}\begin{shaded}
          \begin{proof}
The GLS estimator can be written as :
\begin{align*}
  \widehat{\bm{\beta} }_{GLS} &= \left( \bm{X}^\prime \bm{\Omega}^{-1} \bm{X}   \right)^{-1} \left( \bm{X}^\prime \bm{\Omega}^{-1} \bm{y}    \right) \\
  &= \left( \bm{W}^\prime \bm{X}   \right)^{-1} \bm{W}^\prime \bm{y}  
\end{align*}
where $\bm{\Omega} = \operatorname{diag}\left( \sigma^2_{1},\dots,\sigma^2_{n} \right) $ and $\bm{W} = \bm{\Omega}^{-1} \bm{X}   $.
           \end{proof}\vspace{-2.5em}

       \end{shaded}} 


{\colorlet{shadecolor}{blue!10}\begin{shaded}\vspace{-2em}
    \section*{\large \text{12.3}}   \vspace{-1.5em}
Take the linear model \[
  \bm{y} = \bm{X}\bm{\beta} + \bm{e}    
\]
Let the OLS estimator for $\bm{\beta} $  be $\widehat{\bm{\beta} } $ and the OLS residual be $\widehat{e} = y - \bm{X}\widehat{\bm{\beta} }   $.

Let the IV estimator for $\bm{\beta} $ using some instrument $\bm{Z} $ be $\widetilde{\bm{\beta} } $ and the 
IV residual be $\widetilde{\bm{e} } = \bm{y} - \bm{X}\widetilde{\bm{\beta} }.    $ If $\bm{X} $ is indeed endogenous, will IV ``fit'' better than OLS, in the sense that $\widetilde{\bm{e} }^{\,^\prime} \widetilde{\bm{e} } < \widehat{\bm{e} }^{\,^\prime} \widehat{\bm{e} }?    $
\end{shaded}}
\vspace{-1.5em}

       {\colorlet{shadecolor}{gray!15}\begin{shaded}
          \begin{proof}
            Define $\widetilde{\bm{M} } = \bm{I} - \bm{X}\left( \bm{X}^\prime \bm{P} _{\bm{Z} } \bm{X}   \right)^{-1} \bm{X}^\prime \bm{P} _{\bm{Z} } \bm{y}     $, where $\bm{P} _{\bm{Z} } = \bm{Z}\left( \bm{Z}^\prime \bm{Z}   \right)^{-1} \bm{Z}^\prime  $.

            Then $\widetilde{\bm{e} } = \widetilde{\bm{M} }\bm{e}   $, and the proof is similar to that in Problem 6 of Midterm. Observing $\widetilde{\bm{M} }^\prime \bm{M}\widetilde{\bm{M} } = \bm{M}    $, it is easy to show \[
              \widetilde{\bm{e} }^{\,^\prime} \widetilde{\bm{e} } \geqslant \widehat{\bm{e} }^{\,^\prime} \widehat{\bm{e} }   
            \]
           \end{proof}\vspace{-2.5em}

       \end{shaded}} 

\end{spacing}
\end{document}
