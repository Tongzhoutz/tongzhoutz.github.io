%
% This is the LaTeX template file for lecture notes for CS267,
% Applications of Parallel Computing.  When preparing 
% LaTeX notes for this class, please use this template.
%
% To familiarize yourself with this template, the body contains
% some examples of its use.  Look them over.  Then you can
% run LaTeX on this file.  After you have LaTeXed this file then
% you can look over the result either by printing it out with
% dvips or using xdvi.
%

\documentclass{article}
\setlength{\oddsidemargin}{0.25 in}
\setlength{\evensidemargin}{-0.25 in}
\setlength{\topmargin}{-0.6 in}
\setlength{\textwidth}{6.5 in}
\setlength{\textheight}{8.5 in}
\setlength{\headsep}{0.75 in}
\setlength{\parindent}{0 in}
\setlength{\parskip}{0.1 in}
\usepackage{array}
\usepackage{booktabs} %调整表格线与上下内容的间隔
\usepackage{multirow}
\usepackage{enumitem}
\usepackage{supertabular}
% ADD PACKAGES here:
%

\usepackage{amsmath,amsfonts,graphicx}

%
% The following commands set up the lecnum (lecture number)
% counter and make various numbering schemes work relative
% to the lecture number.
%
\newcounter{lecnum}
%\renewcommand{\thepage}{\thelecnum-\arabic{page}}
\renewcommand{\thesection}{\thelecnum.\arabic{section}}
\renewcommand{\theequation}{\thelecnum.\arabic{equation}}
\renewcommand{\thefigure}{\thelecnum.\arabic{figure}}
\renewcommand{\thetable}{\thelecnum.\arabic{table}}

%
% The following macro is used to generate the header.
%
\newcommand{\lecture}[4]{
   \pagestyle{myheadings}
   \thispagestyle{plain}
   \newpage
   \setcounter{lecnum}{#1}
   \setcounter{page}{1}
   \noindent
   \begin{center}

    \vspace*{4mm}
}
%
% Convention for citations is authors' initials followed by the year.
% For example, to cite a paper by Leighton and Maggs you would type
% \cite{LM89}, and to cite a paper by Strassen you would type \cite{S69}.
% (To avoid bibliography problems, for now we redefine the \cite command.)
% Also commands that create a suitable format for the reference list.
\renewcommand{\cite}[1]{[#1]}
\def\beginrefs{\begin{list}%
        {[\arabic{equation}]}{\usecounter{equation}
         \setlength{\leftmargin}{2.0truecm}\setlength{\labelsep}{0.4truecm}%
         \setlength{\labelwidth}{1.6truecm}}}
\def\endrefs{\end{list}}
\def\bibentry#1{\item[\hbox{[#1]}]}

%Use this command for a figure; it puts a figure in wherever you want it.
%usage: \fig{NUMBER}{SPACE-IN-INCHES}{CAPTION}
\newcommand{\fig}[3]{
			\vspace{#2}
			\begin{center}
			Figure \thelecnum.#1:~#3
			\end{center}
	}
% Use these for theorems, lemmas, proofs, etc.
\newtheorem{theorem}{Theorem}[lecnum]
\newtheorem{lemma}[theorem]{Lemma}
\newtheorem{proposition}[theorem]{Proposition}
\newtheorem{claim}[theorem]{Claim}
\newtheorem{corollary}[theorem]{Corollary}
\newtheorem{definition}[theorem]{Definition}
\newenvironment{proof}{{\bf Proof:}}{\hfill\rule{2mm}{2mm}}

% **** IF YOU WANT TO DEFINE ADDITIONAL MACROS FOR YOURSELF, PUT THEM HERE:

\newcommand\E{\mathbb{E}}
\usepackage{colortbl}
\usepackage{longtable}
\begin{document}
\begin{longtable}[b]{|p{10cm}<{\raggedright}|p{7cm}<{\raggedright}|}
\multicolumn{2}{l}{\small{\textbf{Paper}}} \tabularnewline
% title hei,e
\multicolumn{2}{l}{\small{\color{blue} \large Nonparametric Identification of Dynamic Models with Unobserved State Variables}} \tabularnewline \specialrule{0.15em}{3pt}{3pt}
% Summary 
\bf{\large Primary} & \bf{\large Comments}\\\specialrule{0.15em}{3pt}{3pt}
This paper proposes a novel method for identifying a hidden Markov process: & \\
\begin{itemize}
  \item Only 5 observations are needed in non-stationary cases, while only 4 are enough in station cases.
  \item $(W_{t},X_{t}^{*})$ jointly evolves.
  \item After the Markov kernel is identified, other relevant quantities can be recovered:
    
    \textbf{Markov Kernel = CCP*State Law of Motion}
  \item Application: dynamic optimization models with unobserved process.
  \item Strength:
    \begin{itemize}
      \item Allow time-varying unobserved 
      \item Evolve depending on past values of observables.
    \end{itemize}
\end{itemize}
&    
\vspace{1em}
\emph{How to identify other relevant quantities? Which formulae can ilustrate eqaution (1)}
\bigskip 

Why CCP and SLOM can be recovered?
\vspace{7em}

 \emph{See Arellano Bonhemme 2017 Review Paper, where more applications and examples are discussed.}
\\\specialrule{0.15em}{3pt}{3pt}
%% Assumptions
\bf{\large Model}  &\bf{\large Comments}\\\specialrule{0.15em}{3pt}{3pt}

\begin{itemize}
  \item Observables: two components: action(decision) and state.
  \item Eq. 2 and 3.
  \item Eq. 7.
\end{itemize}
$\begin{aligned} f_{X,Y,Z,S} = \int f_{X | X^{*}, S} f_{X^{*},Z,S}f_{Y|X^{*},Z}\,\mathrm{d}x^{*}, \end{aligned}$

$\begin{aligned}
  f_{W_{t+1}, W_{t}, W_{t-1}, W_{t-2}}= \int f_{W_{t+1}|W_{t},X_{t}^{*}}f_{W_{t}|W_{t-1},X_{t}^{*}} f_{X_{t}^{*},W_{t-1},W_{t-2}}\,\mathrm{d}x_{t}^{*} \\
  = \int f_{W_{t+1}|W_{t},X_{t}^{*}} f_{W_{t},W_{t-1},X_{t}^{*}}f_{W_{t-2}|X_{t}^{*},W_{t-1}}\,\mathrm{d}x_{t}^{*}
\end{aligned}$
\\\specialrule{0.15em}{3pt}{3pt}
%% Models
\bf{\large Assumptions} & \bf{\large Comments} \\ \specialrule{0.15em}{3pt}{3pt}
A1.1. First-order Markovian

A1.2. Limited feedback

   A2. Invertibility. Three injective linear operators.

   A3. Uniqueness of decomposition.

  A4. Monotonicity and Normalization.

  Other assumptions:
  1. $X_{t}$ is scalar and continuous.

  2. $V_{t} \equiv g_{t}(W_{t}).$ 

  3. How to connect Carrol's assumptions with those in this paper?
  \\\specialrule{0.15em}{3pt}{3pt}
%% Lemmas 
\bf{\large Lemmas} & \bf{\large Comments} \\ \specialrule{0.15em}{3pt}{3pt}
\begin{enumerate}[label=$\ast$]
  \item Lemma 1: Representation of the observed density.
  \item Lemma 2: Representation of the Markov Law of Motion.
  \item Lemma 3: Identification of $f_{V_{t+1}| W_{t}, X_{t}^{*}}$.
\end{enumerate}

\\\specialrule{0.15em}{3pt}{3pt}
%% Identification 
\bf{\large Identification Strategy} & \bf{\large Comments} \\ \specialrule{0.15em}{3pt}{3pt}
\begin{enumerate}
  \item Based on \emph{Hu \& Schennach 2008} and \emph{Carrol 2010}.
  \item Unique spectral decomposition: A1 $   \to $ A4.
  \item Two step Identification:
    \begin{enumerate}
      \item By A1 $   \to $ A4, $f_{V_{t+1}|W_{t},X_{t}^{*}}$ identified.
      \item By Lemma 2, the Markov kernel is identified.
      \item Identify the joint distribution of the initial condition: $f_{W_{t-1}},X_{t}^{*}$.
    \end{enumerate}
\end{enumerate}
\end{longtable}
\end{document}





